\documentclass[a4paper,UTF8]{article}
\usepackage{ctex}
\usepackage[margin=1.25in]{geometry}
\usepackage{color}
\usepackage{graphicx}
\usepackage{amssymb}
\usepackage{amsmath}
\usepackage{amsthm}
%\usepackage[thmmarks, amsmath, thref]{ntheorem}
\theoremstyle{definition}
\newtheorem*{solution}{Solution}
\newtheorem*{prove}{Proof}
\usepackage{multirow}
\usepackage{url}
\CTEXoptions[today=old]
%--

%--
\begin{document}
\title{Homework 4}
\author{123456, Mrs Puff}
\maketitle

\section*{Deadline}
Due: October 28, 2019, 23:59. Good luck!

\section*{Problem 1}
Determine if the following formulas are wffs.
\begin{enumerate}
\item $(\forall x)( P(x)\wedge R(x) )\rightarrow( (\forall x)P(x)\wedge Q(x) )$
\item $ (\forall x)( P(x)\leftrightarrow Q(x) )\wedge(\exists x)R(x)\wedge S(x) $
\item $ (\exists x)( (\forall y)P(y)\rightarrow Q(x,y) ) $
\item $ (\exists x)(\exists y)( P(x,y,z)\rightarrow S(u,v) ) $
\item $ ( \forall x )P(x,y)\wedge Q(z) $
\end{enumerate}
\begin{solution}
~\\
Your solution here.
\end{solution}

\section*{Problem 2}
In the following formula, what are bound variables and what are free variables? what is the scope of every quantifier?
\[
(\forall x)( P(x)\wedge Q(x) )\rightarrow( (\forall x)R(x)\wedge Q(z) )
\]
\begin{solution}
~\\
Your solution here.
\end{solution}

\section*{Problem 3}
Represent the following statements with predicate logic formulas. The default domain of discourse includes everything.
\begin{enumerate}
\item For every two different points on a plane, there is and only one straight line can pass through them.
\item People working in Shanghai may not live in Shanghai.
\item If it will be fine tomorrow, some students will go swimming.
\end{enumerate}
\begin{solution}
~\\
Your solution here.
\end{solution}


\section*{Problem 4}
I(x) means x is a positive integer. P(x) means x is a rational number. What's the meaning of $ (\forall x)( I(x)\rightarrow P(x) )\wedge\neg(\forall x)( P(x)\rightarrow I(x) ) $.
\begin{solution}
~\\
Your solution here.
\end{solution}

\section*{Problem 5}
\begin{enumerate}
\item Write a predicate logic formula. If the domain of discourse is \{1,2\}, the formula is satisfiable. If the domain of discourse is \{1\}, the formula is unsatisfiable.
\item If the domain of discourse is \{a,b,c\}, rewrite $(\exists x)(\forall y)P(x,y)$ and $(\forall x)P(x)\rightarrow (\exists y)Q(y)$ like that in 4.5.2 of the textbook.
\item The domain is \{a,b\}. P(a,a)=T, P(a,b)=F, P(b,a)=F and P(b,b)=T. Determine the truth value of the following formulas:
\begin{align*}
&(\forall x)(\exists y)P(x,y)\\
&(\exists x)(\forall y)P(x,y)\\
&(\forall x)(\forall y)P(x,y)\\
&(\exists x)(\exists y)P(x,y)\\
&(\exists y)\neg P(a,y)\\
&(\exists y)(\forall x)P(x,y)\\
\end{align*}
\end{enumerate}

\begin{solution}
~\\
Your solution here.
\end{solution}

\section*{Problem 6}
\begin{enumerate}
\item Convert the following formulas to PNF according to 5.3.1 in the textbook.
\begin{align*}
(\forall x)( P(x)\rightarrow(\exists y)Q(x,y) )\\
(\exists x)P(x,y)\leftrightarrow(\forall z)Q(z)\\
\end{align*}
\item Convert the following formulas to Skolem normal form according to 5.3.2 in the textbook.
\begin{align*}
(\forall x)( P(x)\rightarrow(\exists y)Q(x,y) )\vee(\forall z)R(z)\\
(\exists y)(\forall x)(\forall z)(\exists u)(\forall v)P(x,y,z,u,v)
\end{align*}
\end{enumerate}

\begin{solution}
~\\
Your solution here.
\end{solution}

\section*{Problem 7}
If we use GSAT to solve $(P\vee\neg Q)\wedge(\neg P\vee Q)\wedge(P\vee\neg Q)\wedge\neg P\wedge\neg Q $, give all the starting points that make GSAT return unknown. In this problem, when will GSAT return unsat? When will GSAT return sat?
\begin{solution}
~\\
Your solution here.
\end{solution}

\section*{Problem 8}
Use DPLL to solve this problem. When splitting on decision literals, follow this initial order: 1 false.
\begin{align*}
\emptyset&\parallel~~1\vee\overline{2},~~\overline{1}\vee\overline{2},~~2\vee3,~~\overline{3}\vee4,~~1\vee\overline{4}\\
\end{align*}

\begin{solution}
~\\
Your solution here.
\end{solution}



\end{document}