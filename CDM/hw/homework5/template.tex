\documentclass[a4paper,UTF8]{article}
\usepackage{ctex}
\usepackage[margin=1.25in]{geometry}
\usepackage{color}
\usepackage{graphicx}
\usepackage{amssymb}
\usepackage{amsmath}
\usepackage{amsthm}
%\usepackage[thmmarks, amsmath, thref]{ntheorem}
\theoremstyle{definition}
\newtheorem*{solution}{Solution}
\newtheorem*{prove}{Proof}
\usepackage{multirow}
\usepackage{url}
\CTEXoptions[today=old]
%--

%--
\begin{document}
\title{Homework 5}
\author{123456,蟹老板}
\maketitle


\section*{Deadline}
Due: November 1, 2019, 23:59. Good luck!

\section*{Problem 1}
Prove the following equations and tautological implication. (Hint:5.1, 5.2, 5.4.2 in the textbook)
\begin{enumerate}
\item
\begin{align*}
&\neg(\exists x)(\exists y)( P(x)\wedge P(y)\wedge Q(x)\wedge Q(y)\wedge R(x,y) )\\
=&(\forall x)(\forall y)( ( P(x)\wedge P(y)\wedge Q(x)\wedge Q(y) )\rightarrow \neg R(x,y) )
\end{align*}
\item
\begin{align*}
&(\exists x)( P(x)\rightarrow Q(x) )\\
=&(\forall x)P(x)\rightarrow (\exists x)Q(x)
\end{align*}
\item
\begin{align*}
&(\forall y)(\exists x)( (P(x)\rightarrow q)\vee S(y) )\\
=&(( \forall x )P(x)\rightarrow q )\vee(\forall y)S(y)
\end{align*}
\item
\begin{align*}
&(\exists x)P(x)\rightarrow (\forall x)Q(x)\\
\Rightarrow&(\forall x)( P(x)\rightarrow Q(x) )
\end{align*}
\end{enumerate}
\begin{solution}
~\\
Your solution here.
\end{solution}



\section*{Problem 2}
\begin{enumerate}
\item Prove $(\forall x)( P(x)\vee Q(x) )\wedge (\forall x)( Q(x)\rightarrow\neg R(x) )\Rightarrow(\exists x)( R(x)\rightarrow P(x) )$ by deduction in 5.5.
\item Every student in the university is either an undergraduate or a postgraduate. Some students are male. John is not a postgraduate but he is male. Therefore, if John is a student in the university, he must be an undergraduate. Represent these statements in predicate logic and prove the conclusion ("if John is a student in the university, he must be an undergraduate") by resolution method in 5.6
\end{enumerate} 
\begin{solution}
~\\
Your solution here.
\end{solution}


\section*{Problem 3}
Determine if the following deduction are right. Explain your reasons if the deduction is wrong.
\begin{enumerate}
\item If $(\forall x)P(x)\rightarrow Q(x)$, then $P(a)\rightarrow Q(a)$.
\item If $\neg(\exists x)( \neg P(x)\wedge\neg Q(x) )$, then $\neg( (\exists x)\neg P(x)\wedge (\exists x)\neg Q(x) )$.
\item If $(\exists x)( \neg P(x)\wedge\neg Q(x) )$, then $(\exists x)\neg P(x)\wedge (\exists x)\neg Q(x)$.
\end{enumerate}
\begin{solution}
~\\
Your solution here.
\end{solution}


\section*{Problem 4}
Determine if the following formulas are universally valid. If they are universally valid, give the proof. Otherwise, give the counterexample.
\begin{enumerate}
\item $( (\exists x)P(x)\rightarrow (\exists x)Q(x) )\rightarrow(\exists x)( P(x)\rightarrow Q(x) )$
\item $(\forall x)(\exists y)P(x,y)\rightarrow (\exists y)(\forall x)P(x,y) $
\end{enumerate}
\begin{solution}
~\\
Your solution here.
\end{solution}


\section*{Problem 5}
Encode $f(b)=b, f(b)=f(a), a=c,  f(c)\neq b$ into a SAT problem and determine the satisfiability of it. You don't need to write the process of DPLL. Directly write its satisfiability.
\begin{solution}
~\\
Your solution here.
\end{solution}


\section*{Problem 6}
Describe how to solve $f(b)=b, f(b)=f(a), a=c, b\neq c, f(c)\neq a$ by EUF solver. Draw the graph like that in slides.
\begin{solution}
~\\
Your solution here.
\end{solution}


\section*{Problem 7}
Solve $ f(i) - f(j)\neq 0\wedge i-j= 0 $ by the Nelson-Oppen Method.
\begin{solution}
~\\
Your solution here.
\end{solution}

\end{document}

